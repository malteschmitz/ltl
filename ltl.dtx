% \iffalse meta-comment
%
% Copyright (C) 2013 by Malte Schmitz <malte@schmitz-sh.de>
% ---------------------------------------------------------------------------
% This work may be distributed and/or modified under the
% conditions of the LaTeX Project Public License, either version 1.3
% of this license or (at your option) any later version.
% The latest version of this license is in
%   http://www.latex-project.org/lppl.txt
% and version 1.3 or later is part of all distributions of LaTeX
% version 2005/12/01 or later.
%
% This work has the LPPL maintenance status `maintained'.
%
% The Current Maintainer of this work is Malte Schmitz.
%
% This work consists of the files ltl.dtx and ltl.ins
% and the derived filebase ltl.sty.
%
% \fi
%
% \iffalse
%<*driver>
\ProvidesFile{ltl.dtx}
%</driver>
%<package>\NeedsTeXFormat{LaTeX2e}[1999/12/01]
%<package>\ProvidesPackage{ltl}
%<*package>
    [2013/03/05 v0.2 configurable LTL math operators]
%</package>
%
%<*driver>
\documentclass{ltxdoc}
\setlength{\parskip}{.5\baselineskip}
\setlength{\parindent}{0pt}
\usepackage{fontenc}
\usepackage{lmodern}
\usepackage{ltl}
\usepackage{hyperref}
\hypersetup{breaklinks=true,
            pdfborder={0 0 0},
            pdfhighlight={/N},
            pdftitle={Configurable LTL math operators with the ltl package},
            pdfauthor={Malte Schmitz}}
\usepackage{paralist}
\EnableCrossrefs
\CodelineIndex
\RecordChanges
\begin{document}
  \DocInput{ltl.dtx}
  \PrintChanges
  \PrintIndex
\end{document}
%</driver>
% \fi
%
% \CheckSum{0}
%
% \CharacterTable
%  {Upper-case    \A\B\C\D\E\F\G\H\I\J\K\L\M\N\O\P\Q\R\S\T\U\V\W\X\Y\Z
%   Lower-case    \a\b\c\d\e\f\g\h\i\j\k\l\m\n\o\p\q\r\s\t\u\v\w\x\y\z
%   Digits        \0\1\2\3\4\5\6\7\8\9
%   Exclamation   \!     Double quote  \"     Hash (number) \#
%   Dollar        \$     Percent       \%     Ampersand     \&
%   Acute accent  \'     Left paren    \(     Right paren   \)
%   Asterisk      \*     Plus          \+     Comma         \,
%   Minus         \-     Point         \.     Solidus       \/
%   Colon         \:     Semicolon     \;     Less than     \<
%   Equals        \=     Greater than  \>     Question mark \?
%   Commercial at \@     Left bracket  \[     Backslash     \\
%   Right bracket \]     Circumflex    \^     Underscore    \_
%   Grave accent  \`     Left brace    \{     Vertical bar  \|
%   Right brace   \}     Tilde         \~}
%
%
% \changes{v0.1}{2012/10/05}{Initial version}
% \changes{v0.2}{2013/03/05}{Macro collection transformed into a \LaTeX\
%     package.}
%
% \DoNotIndex{\providecommand,\operatorname,\let}
%
% \GetFileInfo{ltl.dtx}
% \title{Configurable LTL math operators with the \textsf{ltl}
%   package\thanks{This document
%   corresponds to \textsf{ltl}~\fileversion, dated~\filedate.}}
% \author{Malte Schmitz \\ \url{malte@schmitz-sh.de}}
%
% \maketitle
%
% \begin{abstract}
%   The \textsf{ltl} package contains a set of macros for typesetting operators
%   used in formulas of the linear temporal logic (LTL) in the Manna/Pnueli
%   or modern character based notation. This package provides options to switch
%   between characters and symbols as operators. As a backend for the symbols
%   either the \textsf{ltlfonts} by Matteo Slanina or Ti\emph{k}Z drawings
%   provided in this package can be used.
% \end{abstract}
%
% \section{Introduction}
%
% Put text here.
%
% \section{Usage}
%
% Put text here.
%
% \section{Showcase}
% 
% \subsection{Direct Symbol Usage}
% circle: $\LTLcircle x$ \\
% circle with minus: $\LTLcircleminus x$ \\
% circle with tilde: $\LTLcircletilde x$ \\
% diamond: $\LTLdiamond x$ \\
% diamond with minus: $\LTLdiamondminus x$ \\
% square: $\LTLsquare x$ \\
% square with minus: $\LTLsquareminus x$
%
% \subsection{Semantic Interface Usage with symbols and without weakindex}
% until: $x \LTLuntil x$ \\
% weak until: $x \LTLweakuntil x$ \\
% release: $x \LTLrelease x$ \\
% since: $x \LTLsince x$ \\
% weak since: $x \LTLweaksince x$ (back: $x \LTLback x$) \\
% globally: $\LTLglobally x$ \\
% finally: $\LTLfinally x$ (eventually: $\LTLeventually x$) \\
% next: $\LTLnext x$ \\
% weak next: $\LTLweaknext x$ \\
% previous: $\LTLprevious x$ (prev: $\LTLprev x$) \\
% weak previous: $\LTLweakprevious x$ (weak prev: $\LTLweakprev x$) \\
% past globally: $\LTLpastglobally x$ \\
% past finally: $\LTLpastfinally x$ (once: $\LTLonce x$) \\
% predict: $\LTLpredict x$ \\
% record: $\LTLrecord x$ \\
% stop: $\LTLstop x$ \\
% implication: $x \LTLimplication x$ (imp: $x \LTLimp x$) \\
% equal: $x \LTLequivalent x$ (equ: $x \LTLequ x$) \\
% and: $x \LTLand x$ \\
% or: $x \LTLor x$ \\
% false: $\LTLfalse$ \\
% true: $\LTLtrue$
%
% \subsection{Semantic Interface Usage with symbols and with weakindex}
% \ltlsetup{weakindex}
% until: $x \LTLuntil x$ \\
% weak until: $x \LTLweakuntil x$ \\
% release: $x \LTLrelease x$ \\
% since: $x \LTLsince x$ \\
% weak since: $x \LTLweaksince x$ (back: $x \LTLback x$) \\
% globally: $\LTLglobally x$ \\
% finally: $\LTLfinally x$ (eventually: $\LTLeventually x$) \\
% next: $\LTLnext x$ \\
% weak next: $\LTLweaknext x$ \\
% previous: $\LTLprevious x$ (prev: $\LTLprev x$) \\
% weak previous: $\LTLweakprevious x$ (weak prev: $\LTLweakprev x$) \\
% past globally: $\LTLpastglobally x$ \\
% past finally: $\LTLpastfinally x$ (once: $\LTLonce x$) \\
% predict: $\LTLpredict x$ \\
% record: $\LTLrecord x$ \\
% stop: $\LTLstop x$ \\
% implication: $x \LTLimplication x$ (imp: $x \LTLimp x$) \\
% equal: $x \LTLequivalent x$ (equ: $x \LTLequ x$) \\
% and: $x \LTLand x$ \\
% or: $x \LTLor x$ \\
% false: $\LTLfalse$ \\
% true: $\LTLtrue$
%
% \subsection{Semantic Interface Usage with characters and without weakindex}
% \ltlsetup{characters}
% until: $x \LTLuntil x$ \\
% weak until: $x \LTLweakuntil x$ \\
% release: $x \LTLrelease x$ \\
% since: $x \LTLsince x$ \\
% weak since: $x \LTLweaksince x$ (back: $x \LTLback x$) \\
% globally: $\LTLglobally x$ \\
% finally: $\LTLfinally x$ (eventually: $\LTLeventually x$) \\
% next: $\LTLnext x$ \\
% weak next: $\LTLweaknext x$ \\
% previous: $\LTLprevious x$ (prev: $\LTLprev x$) \\
% weak previous: $\LTLweakprevious x$ (weak prev: $\LTLweakprev x$) \\
% past globally: $\LTLpastglobally x$ \\
% past finally: $\LTLpastfinally x$ (once: $\LTLonce x$) \\
% predict: $\LTLpredict x$ \\
% record: $\LTLrecord x$ \\
% stop: $\LTLstop x$ \\
% implication: $x \LTLimplication x$ (imp: $x \LTLimp x$) \\
% equal: $x \LTLequivalent x$ (equ: $x \LTLequ x$) \\
% and: $x \LTLand x$ \\
% or: $x \LTLor x$ \\
% false: $\LTLfalse$ \\
% true: $\LTLtrue$
%
% \subsection{Semantic Interface Usage with characters and with weakindex}
% \ltlsetup{characters,weakindex}
% until: $x \LTLuntil x$ \\
% weak until: $x \LTLweakuntil x$ \\
% release: $x \LTLrelease x$ \\
% since: $x \LTLsince x$ \\
% weak since: $x \LTLweaksince x$ (back: $x \LTLback x$) \\
% globally: $\LTLglobally x$ \\
% finally: $\LTLfinally x$ (eventually: $\LTLeventually x$) \\
% next: $\LTLnext x$ \\
% weak next: $\LTLweaknext x$ \\
% previous: $\LTLprevious x$ (prev: $\LTLprev x$) \\
% weak previous: $\LTLweakprevious x$ (weak prev: $\LTLweakprev x$) \\
% past globally: $\LTLpastglobally x$ \\
% past finally: $\LTLpastfinally x$ (once: $\LTLonce x$) \\
% predict: $\LTLpredict x$ \\
% record: $\LTLrecord x$ \\
% stop: $\LTLstop x$ \\
% implication: $x \LTLimplication x$ (imp: $x \LTLimp x$) \\
% equal: $x \LTLequivalent x$ (equ: $x \LTLequ x$) \\
% and: $x \LTLand x$ \\
% or: $x \LTLor x$ \\
% false: $\LTLfalse$ \\
% true: $\LTLtrue$
%
% \section{Installing \textsf{ltlfonts}}
% 
% \textsf{ltlfonts} is a free font developed by Matteo Slanina
% containing mathematical symbols for typesetting
% formulas of linear temporal logic (LTL) in the Manna/Pnueli notation.
%
% If \textsf{ltlfonts} is loaded this package will use the symbols of LTLFonts
% instead of the TikZ drawings provided in this package to define its macros.
%
% \textsf{ltlfonts} can be downloaded from
% \url{http://theory.stanford.edu/~matteo/ltlfonts/}
%
% To use it with TeX, LaTeX and dvips you can just copy all the files from
% the zip file (ignoring all folders) next to your tex file and run
% \begin{verbatim}
% latex yourfile.tex
% dvips -u +ltlfonts.map yourfile.tex</pre>
% \end{verbatim}
%
% To use it with PdfTeX or PdfLaTeX you have to install the font into your LaTeX
% system first. To do so follow these steps:
%
% Find your local TeX tree (texmf-local). In TeX Live and MaxTeX the default
% is |/usr/local/texlive/texmf-local|. In MicTeX the default is
% |%APPDATA%\MikTeX\2.9|
%
% Copy these files to the given directories:
% \begin{compactitem}[--]
%   \item |fonts/afm/matteo/ltlfonts/ltlfonts.afm|
%   \item |fonts/map/dvips/matteo/ltlfonts.map|
%   \item |fonts/tfm/matteo/ltlfonts/ltlfonts.tfm|
%   \item |fonts/type1/matteo/ltlfonts/ltlfonts.pfb|
%   \item |fonts/type1/matteo/ltlfonts/ltlfonts.pfm|
%   \item |tex/latex/ltlfonts/ltlfonts.sty|
%   \item |tex/latex/ltlfonts/Ultlfonts.fd|
% \end{compactitem}
% 
% After getting your new files into their proper location,
% you must update the so-called ``TeX filename database''.
% \begin{compactitem}[--]
%   \item on MikTeX run |initexmf --update-fndb|
%   \item on MacTeX run |mktexlsr|
%   \item on TeXLive run |mktexlsr|
% \end{compactitem}
% 
% After recording the new files, the last step is to update so-called ``map''
% files with the information about the new font.
%
% On MikTex run |initexmf --edit-config-file updmap|. A configuration file gets
% opened in your default editor. Add the line
% \begin{verbatim}
% Map ltlfonts.map
% \end{verbatim}
% (Don't worry if the file is initially empty.)
%
% On TeX Live and MacTeX run
% \begin{verbatim}
% updmap-sys --enable Map=ltlfonts.map
% \end{verbatim}
%
% \textsf{ltlfonts} is now installed and can be used. As a test you can compile
% this LaTeX code:
%
% \begin{verbatim}
% \documentclass{article}
% \usepackage{ltlfonts}
% \begin{document}
% LTLFonts provides some nice boxes and circles. For example:
% \[ \LTLsquare \LTLdiamond \LTLcircle a, \LTLcircleminus
% \LTLdiamondminus \LTLsquareminus \LTLcircletilde b, \LTLsquarehat
% \LTLdiamondminushat c \]
% \end{document}
% \end{verbatim}
%
% The \textsf{ltlfonts} package provides a set of LTL symbols. It does not
% provide any semantically named commands and it does not allow switching to
% characeter based LTL symbols. It just lets you use the LTL symbols of the
% LTLFonts font in a \LaTeX\ document.
%
%
% \StopEventually{}
%
% \section{Implementation}
%
% \iffalse
%<*package>
% \fi
%
% We start loading some required packages. \textsf{xkeyval} is used to handle
% the options. \textsf{amsmath} provides |\operatorname| which handles
% the spacing around the declared operators automatically, \textsf{amssymb}
% provides some additional symbols some operator macros are based on and
% Ti\emph{k}Z is used to draw the LTL symbols.
%    \begin{macrocode}
\RequirePackage{xkeyval}
\RequirePackage{amsmath}
\RequirePackage{amssymb}
\RequirePackage{tikz}
%    \end{macrocode}
%
% \subsection{Ti\emph{k}Z symbol operators}
%
% We now define the symbols using Ti\emph{k}Z. The macros are named after the
% macros defined by the LTLFonts package. Every macro is only defined unless
% it is already defined. This way we use the macros of the LTLFonts package
% if it is loaded. The |\tikz| command with |baseline| option is used to create
% ``inline'' graphics as this technique called in the manual:
% \begin{quote}
%   Normally, the lower end of the picture is put on the baseline of the
%   surrounding text. Using this option, you can specify that the picture
%   should be raised or lowered such that the given height is on the baseline.
% \end{quote}
% The value defaults to |0pt|.
%
% \begin{macro}{LTLcircle}
% Draws a circle, e.g. for the LTL next operator.
%    \begin{macrocode}
\providecommand{\LTLcircle}{\operatorname{%
  \tikz[baseline]{
    \draw[line width=.12ex]
      (0,.6ex) circle (.8ex);
  }}}{}
%    \end{macrocode}
% \end{macro}
%
% \begin{macro}{LTLcircleminus}
% Draws a circle with minus in it, e.g. for LTL previous resp. past next
% operator.
%    \begin{macrocode}
\providecommand{\LTLcircleminus}{\operatorname{%
  \tikz[baseline]{
    \draw[line width=.12ex]
      (0,.6ex) circle (.8ex);
    \draw[line width=.09ex,line cap=round]
      (-.4ex,.6ex) -- (.4ex,.6ex);
  }}}{}
%    \end{macrocode}
% \end{macro}
%
% \begin{macro}{LTLcircletilde}
% Draw a circle with tilde in it, e.g. for LTL weak previous resp weak past next
% operator.
%    \begin{macrocode}
\providecommand{\LTLcircletilde}{\operatorname{%
  \tikz[baseline]{
    \draw[line width=.12ex]
      (0,.6ex) circle (.8ex);
    \draw[line width=.09ex,line cap=round,rounded corners=0.2ex]
      (-.4ex,.55ex) -- (-.2ex,.8ex) -- (.2ex,.4ex) -- (.4ex,.65ex);
  }}}{}
%    \end{macrocode}
% \end{macro}
%
% \begin{macro}{LTLdiamond}
%   Draws diamond, e.g. for the LTL finally resp. eventually operator.
%    \begin{macrocode}
\providecommand{\LTLdiamond}{\operatorname{%
  \tikz[baseline]{
    \draw[line width=.12ex,line join=round]
      (0ex,.6ex) -- (.95ex,1.55ex) -- (1.9ex,.6ex) -- (.95ex,-.35ex) -- cycle;
  }}}{}
%    \end{macrocode}
% \end{macro}
%
% \begin{macro}{LTLdiamondminus}
% Draws a diamond with minus in it, e.g. for the LTL past finally resp. past
% eventually resp. once operator.
%    \begin{macrocode}
\providecommand{\LTLdiamondminus}{\operatorname{%
  \tikz[baseline]{
    \draw[line width=.12ex,line join=round]
      (0ex,.6ex) -- (.95ex,1.55ex) -- (1.9ex,.6ex) -- (.95ex,-.35ex) -- cycle;
    \draw[line width=.09ex,line cap=round]
      (.5ex,.6ex) -- (1.3ex,.6ex);
  }}}{}
%    \end{macrocode}
% \end{macro}
% 
% \begin{macro}{LTLsquare}
%   Draws a square, e.g. for the LTL globally operator.
%    \begin{macrocode}
\providecommand{\LTLsquare}{\operatorname{%
  \tikz[baseline]{
    \draw[line width=.12ex,line join=round]
      (0ex,-.2ex) -- (0ex,1.3ex) -- (1.5ex,1.3ex) -- (1.5ex,.-.2ex) -- cycle;
  }}}{}
%    \end{macrocode}
% \end{macro}
%
% \begin{macro}{LTLsquareminus}
% Draws a square with minus in it, e.g. for the LTL past globally operator.
%    \begin{macrocode}
\providecommand{\LTLsquareminus}{\operatorname{%
  \tikz[baseline]{
    \draw[line width=.12ex,line join=round]
      (0ex,-.2ex) -- (0ex,1.3ex) -- (1.5ex,1.3ex) -- (1.5ex,.-.2ex) -- cycle;
    \draw[line width=.09ex,line cap=round]
      (.35ex,.6ex) -- (1.15ex,.6ex);
  }}}{}
%    \end{macrocode}
% \end{macro}
%
%    \begin{macrocode}
\DeclareMathOperator{\LTLcirclew}{\LTLcircle_w}
\DeclareMathOperator{\LTLcircleminusw}{\LTLcircleminus_w}
%    \end{macrocode}
%
%
% \subsection{Character operators}
%
%    \begin{macrocode}
\DeclareMathOperator{\LTLu}{\mathcal{U}}
\DeclareMathOperator{\LTLuw}{\mathcal{U}_w}
\DeclareMathOperator{\LTLb}{\mathcal{B}}
\DeclareMathOperator{\LTLr}{\mathcal{R}}
\DeclareMathOperator{\LTLs}{\mathcal{S}}
\DeclareMathOperator{\LTLsw}{\mathcal{S}_w}
\DeclareMathOperator{\LTLw}{\mathcal{W}}
\DeclareMathOperator{\LTLf}{\mathcal{F}}
\DeclareMathOperator{\LTLg}{\mathcal{G}}
\DeclareMathOperator{\LTLx}{\mathcal{X}}
\DeclareMathOperator{\LTLxw}{\mathcal{X}_w}  
\DeclareMathOperator{\LTLp}{\mathcal{P}}
\DeclareMathOperator{\LTLpw}{\mathcal{P}_w}
\DeclareMathOperator{\LTLh}{\mathcal{H}}
\DeclareMathOperator{\LTLo}{\mathcal{O}}
%    \end{macrocode}
% \subsection{Semantic Macros for LTL Operators}
%
%   \begin{macrocode}
\providecommand{\LTLuntil}{\LTLu}
\providecommand{\LTLrelease}{\LTLr}
\providecommand{\LTLsince}{\LTLs}

\providecommand{\LTLweakuntil}{\relax}
\providecommand{\LTLweaksince}{\relax}
\providecommand{\LTLglobally}{\relax}
\providecommand{\LTLfinally}{\relax}
\providecommand{\LTLnext}{\relax}
\providecommand{\LTLweaknext}{\relax}
\providecommand{\LTLprevious}{\relax}
\providecommand{\LTLweakprevious}{\relax}
\providecommand{\LTLpastgloabally}{\relax}
\providecommand{\LTLpastfinally}{\relax}

\newcommand{\ltl@define}{%
  \ifKV@ltl@weakindex%
    \let\LTLweakuntil\LTLuw
    \let\LTLweaksince\LTLsw
  \else
    \let\LTLweakuntil\LTLw
    \let\LTLweaksince\LTLb
  \fi
  \ifKV@ltl@characters%
    \let\LTLglobally\LTLg
    \let\LTLfinally\LTLf
    \let\LTLnext\LTLx
    \let\LTLweaknext\LTLxw
    \let\LTLprevious\LTLp
    \let\LTLweakprevious\LTLpw
    \let\LTLpastgloabally\LTLh
    \let\LTLpastfinally\LTLo
  \else
    \let\LTLglobally\LTLsquare
    \let\LTLfinally\LTLdiamond
    \let\LTLnext\LTLcircle
    \let\LTLweaknext\LTLcirclew
    \let\LTLprevious\LTLcircleminus
    \ifKV@ltl@weakindex
      \let\LTLweakprevious\LTLcircleminusw
    \else
      \let\LTLweakprevious\LTLcircletilde
    \fi
    \let\LTLpastglobally\LTLsquareminus
    \let\LTLpastfinally\LTLdiamondminus
  \fi
  \let\LTLeventually\LTLfinally  
  \let\LTLprev\LTLprevious
  \let\LTLweakprev\LTLweakprevious
  \let\LTLonce\LTLpastfinally
  \let\LTLback\LTLweaksince
}
%    \end{macrocode}
%
% \subsection{Option Handling}
%
% We now declare the \textsf{xkeyval} boolean keys. The optional parameter
% |true| is the default value that gets used if you only specify the key with
% a value.
%    \begin{macrocode} 
\define@boolkeys{ltl}{characters,weakindex}[true]
%    \end{macrocode}
%
% This macro processes the keys and values passed by the user to the class
% or package.
%    \begin{macrocode}
\ProcessOptionsX<ltl>
\ltl@define
%    \end{macrocode}
%
% \begin{macro}{ltlsetup}
%    \begin{macrocode}
\newcommand{\ltlsetup}[1]{%
  \setkeys{ltl}{#1}%
  \ltl@define%
}
%    \end{macrocode}
% \end{macro}
%
% \subsection{Further macros}
%
% auxilaries
%    \begin{macrocode}
\DeclareMathOperator{\LTLpredict}{\rhd}
\DeclareMathOperator{\LTLrecord}{\lhd}
\DeclareMathOperator{\LTLstop}{stop}
\DeclareMathOperator{\LTLimplication}{\rightarrow}
\DeclareMathOperator{\LTLequivalent}{\leftrightarrow}
\DeclareMathOperator{\LTLand}{\wedge}
\DeclareMathOperator{\LTLor}{\vee}
\DeclareMathOperator{\LTLfalse}{\bot}
\DeclareMathOperator{\LTLtrue}{\top}
\let\LTLimp\LTLimplication
\let\LTLequ\LTLequivalent
%    \end{macrocode}
%
% \iffalse
%</package>
% \fi
%
% \Finale
\endinput
